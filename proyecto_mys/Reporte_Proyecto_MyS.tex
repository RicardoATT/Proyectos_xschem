\documentclass[conference]{IEEEtran}
\IEEEoverridecommandlockouts
% The preceding line is only needed to identify funding in the first footnote. If that is unneeded, please comment it out.
\usepackage{cite}
\usepackage[spanish]{babel}
\usepackage{amsmath,amssymb,amsfonts}
\usepackage{algorithmic}
\usepackage{graphicx}
\usepackage{textcomp}
\usepackage{xcolor}
\setlength{\parskip}{12pt}
\usepackage[numbers]{natbib}
%\renewcommand{\refname}{Referencias}
\def\BibTeX{{\rm B\kern-.05em{\sc i\kern-.025em b}\kern-.08em
    T\kern-.1667em\lower.7ex\hbox{E}\kern-.125emX}}
\begin{document}

\title{Desarrollo de sinapsis memristiva en una red neuronal pulsante implementada con transistores CMOS del nodo tecnológico SKY130\\
%{\footnotesize \textsuperscript{*}Note: Sub-titles are not captured in Xplore and should not be used}
%\thanks{Identify applicable funding agency here. If none, delete this.}
}

\author{\IEEEauthorblockN{1\textsuperscript{st} Ricardo Aldair Tirado Torres}
\IEEEauthorblockA{\textit{Centro de Investigación en Computación} \\
\textit{Instituto Politécnico Nacional}\\
Ciudad de México, México \\
rtiradot2023@cic.ipn.mx}
\and
\IEEEauthorblockN{2\textsuperscript{nd} Ricardo Barrón Fernández}
\IEEEauthorblockA{\textit{Centro de Investigación en Computación} \\
	\textit{Instituto Politécnico Nacional}\\
	Ciudad de México, México \\
	rbarron@cic.ipn.mx}
\and
\IEEEauthorblockN{3\textsuperscript{rd} Víctor Hugo Ponce Ponce}
\IEEEauthorblockA{\textit{Centro de Investigación en Computación} \\
	\textit{Instituto Politécnico Nacional}\\
	Ciudad de México, México \\
	vponce@cic.ipn.mx}
}

\maketitle

\begin{abstract}
This document is a model and instructions for \LaTeX.
This and the IEEEtran.cls file define the components of your paper [title, text, heads, etc.]. *CRITICAL: Do Not Use Symbols, Special Characters, Footnotes, 
or Math in Paper Title or Abstract.
\end{abstract}

\begin{IEEEkeywords}
Redes neuronales pulsantes, sinapsis memristiva, plasticidad dependiente del tiempo de pulso (STDP), SKY130.
\end{IEEEkeywords}

\section{Introduction}
El cómputo neuromórfico es un campo emergente en la ingeniería y la informática que se inspira en la arquitectura y funcionamiento del cerebro humano para diseñar sistemas de procesamiento de información más eficientes y adaptativos. A diferencia de las computadoras tradicionales, que siguen el modelo de von Neumann, los sistemas neuromórficos emplean una arquitectura distribuida y paralela, emulando las redes neuronales del cerebro. Esta aproximación no solo promete una mayor eficiencia energética y velocidad de procesamiento, sino también una capacidad de aprendizaje y adaptación superior.

En el corazón del cómputo neuromórfico se encuentran las redes neuronales pulsantes, que son modelos de redes neuronales que procesan información en forma de picos de actividad eléctrica, o pulsos. Estos modelos son más biológicamente realistas que las redes neuronales tradicionales utilizadas en el aprendizaje profundo. Las redes neuronales pulsantes permiten un procesamiento temporal y espacial más natural de la información, similar al que ocurre en el cerebro, lo que resulta en un rendimiento superior en tareas que requieren reconocimiento de patrones y procesamiento sensorial en tiempo real.

El aprendizaje dependiente del tiempo de los picos (STDP, por sus siglas en inglés) es un mecanismo clave en las redes neuronales pulsantes. El STDP es una forma de aprendizaje sináptico que ajusta la fuerza de las conexiones entre neuronas en función del momento en que ocurren los pulsos. Si una neurona emisora envía un pulso justo antes de que una neurona receptora envíe el suyo, la conexión entre ambas se fortalece. Este principio, que se observa en las sinapsis biológicas, permite a las redes neuromórficas aprender y adaptarse de manera más eficiente, logrando un aprendizaje no supervisado y dinámico.

Para implementar estos conceptos, se utiliza la tecnología CMOS (Complementary Metal-Oxide-Semiconductor), ampliamente conocida por su uso en la fabricación de chips de computadora tradicionales. Los avances en CMOS permiten la construcción de circuitos neuromórficos que integran neuronas y sinapsis artificiales, capaces de emular la actividad del cerebro humano. La tecnología CMOS ofrece una plataforma robusta y escalable para desarrollar hardware neuromórfico, facilitando la creación de sistemas que son tanto energéticamente eficientes como capaces de realizar cálculos complejos en paralelo.

Ejemplos notables de esta implementación incluyen proyectos como IBM TrueNorth y Intel Loihi, que han desarrollado chips neuromórficos con millones de neuronas y sinapsis artificiales. Estos chips están diseñados para tareas que van desde el reconocimiento de patrones hasta el control de robots, mostrando una eficiencia energética y un rendimiento que superan a las arquitecturas tradicionales. A medida que la tecnología CMOS sigue evolucionando, se espera que las capacidades del cómputo neuromórfico se expandan, abriendo nuevas posibilidades en la inteligencia artificial y más allá.

El cómputo neuromórfico, con sus redes neuronales pulsantes y el aprendizaje STDP, representa una revolución en cómo concebimos y desarrollamos sistemas de procesamiento de información. La integración de estos conceptos en la tecnología CMOS no solo promete un salto significativo en la eficiencia y capacidad de las máquinas, sino que también nos acerca un paso más a replicar la asombrosa complejidad y eficiencia del cerebro humano.

El memristor, un componente eléctrico cuya resistencia varía en función de la cantidad y la dirección de la carga eléctrica que lo atraviesa, ha emergido como una pieza fundamental en el campo del cómputo neuromórfico. Descubierto teóricamente por Leon Chua en 1971 y desarrollado prácticamente en los laboratorios de HP en 2008, el memristor es a menudo considerado el cuarto elemento de los circuitos pasivos junto con el resistor, el capacitor y el inductor. Su capacidad para "recordar" el último estado de resistencia hace que el memristor sea especialmente adecuado para imitar las sinapsis biológicas en sistemas neuromórficos.

En el contexto del cómputo neuromórfico, los memristores se utilizan como sinapsis artificiales que conectan las neuronas en redes neuronales pulsantes. Estas redes, que procesan información en forma de picos de actividad eléctrica similares a los impulsos neuronales del cerebro, se benefician enormemente de la naturaleza adaptable y no volátil del memristor. La resistencia del memristor cambia en respuesta a los picos eléctricos, permitiendo la modulación sináptica de manera analógica, lo cual es crucial para replicar los procesos de aprendizaje y memoria del cerebro humano.

El aprendizaje dependiente del tiempo de los picos (STDP, por sus siglas en inglés) es un mecanismo de plasticidad sináptica que puede ser implementado de manera efectiva utilizando memristores. En STDP, la fuerza de una sinapsis se ajusta en función de la sincronización relativa de los picos pre y post-sinápticos. Si un pulso en la neurona presináptica precede a un pulso en la neurona postsináptica, la sinapsis se fortalece (potenciación). Por el contrario, si el pulso presináptico sigue al postsináptico, la sinapsis se debilita (depresión). Esta regla temporal puede ser implementada directamente en memristores debido a su capacidad para cambiar su resistencia en función de la historia temporal de los pulsos que los atraviesan.

La integración de memristores en la tecnología CMOS (Complementary Metal-Oxide-Semiconductor) ha permitido avances significativos en la construcción de circuitos neuromórficos eficientes y escalables. Los memristores, debido a su tamaño reducido y bajo consumo de energía, se pueden integrar en matrices densas junto con transistores CMOS para crear chips neuromórficos que emulan el funcionamiento del cerebro humano de manera más realista y eficiente. Estos chips, como los desarrollados en proyectos como IBM TrueNorth e Intel Loihi, incorporan miles de millones de sinapsis memristivas capaces de aprendizaje autónomo y procesamiento paralelo, replicando la complejidad y eficiencia del cerebro.

En este trabajo, se realizan las siguientes contribuciones:

\begin{enumerate}
    \item Un circuito de neurona tipo LIF es implementada en el dominio analógico en un nivel de transistores CMOS. 
    \item Una sinapsis del tipo memristiva, que puede cambiar su resistencia en respuesta a las señales eléctricas, lo que permite imitar la plasticidad sináptica del cerebro humano. Esta capacidad de adaptación es crucial para el aprendizaje y la memoria en sistemas neuromórficos.
    \item Una red neuronal sencilla, con una capa de entrada de x neuronas y una capa de salida de y neuronas. El nodo tecnológico empleado es el de SKY130, un proceso de fabricación de código abierto, que reduce costos asociados con el diseño y producción de chips, facilitando la creación de sistemas neuromórficos avanzados.
\end{enumerate}

\section{Sistema propuesto}

\subsection{Neurona LIF}
El modelo de neurona utilizado fue propuesto \cite{Shamsi_2018}.

\subsection{Sinapsis memristiva}
El modelo de memristor utilizado para este proyecto es el proporcionado por 

\section{Operación del sistema}
Before you begin to format your paper, first write and save the content as a 
separate text file. Complete all content and organizational editing before 
formatting. Please note sections \ref{AA}--\ref{SCM} below for more information on 
proofreading, spelling and grammar.

Keep your text and graphic files separate until after the text has been 
formatted and styled. Do not number text heads---{\LaTeX} will do that 
for you.

\section{Resultados de la simulación}
Before you begin to format your paper, first write and save the content as a 
separate text file. Complete all content and organizational editing before 
formatting. Please note sections \ref{AA}--\ref{SCM} below for more information on 
proofreading, spelling and grammar.

Keep your text and graphic files separate until after the text has been 
formatted and styled. Do not number text heads---{\LaTeX} will do that 
for you.

\subsection{Abbreviations and Acronyms}\label{AA}
Define abbreviations and acronyms the first time they are used in the text, 
even after they have been defined in the abstract. Abbreviations such as 
IEEE, SI, MKS, CGS, ac, dc, and rms do not have to be defined. Do not use 
abbreviations in the title or heads unless they are unavoidable.

\subsection{Units}
\begin{itemize}
\item Use either SI (MKS) or CGS as primary units. (SI units are encouraged.) English units may be used as secondary units (in parentheses). An exception would be the use of English units as identifiers in trade, such as ``3.5-inch disk drive''.
\item Avoid combining SI and CGS units, such as current in amperes and magnetic field in oersteds. This often leads to confusion because equations do not balance dimensionally. If you must use mixed units, clearly state the units for each quantity that you use in an equation.
\item Do not mix complete spellings and abbreviations of units: ``Wb/m\textsuperscript{2}'' or ``webers per square meter'', not ``webers/m\textsuperscript{2}''. Spell out units when they appear in text: ``. . . a few henries'', not ``. . . a few H''.
\item Use a zero before decimal points: ``0.25'', not ``.25''. Use ``cm\textsuperscript{3}'', not ``cc''.)
\end{itemize}

\subsection{Equations}
Number equations consecutively. To make your 
equations more compact, you may use the solidus (~/~), the exp function, or 
appropriate exponents. Italicize Roman symbols for quantities and variables, 
but not Greek symbols. Use a long dash rather than a hyphen for a minus 
sign. Punctuate equations with commas or periods when they are part of a 
sentence, as in:
\begin{equation}
a+b=\gamma\label{eq}
\end{equation}

Be sure that the 
symbols in your equation have been defined before or immediately following 
the equation. Use ``\eqref{eq}'', not ``Eq.~\eqref{eq}'' or ``equation \eqref{eq}'', except at 
the beginning of a sentence: ``Equation \eqref{eq} is . . .''

\subsection{\LaTeX-Specific Advice}

Please use ``soft'' (e.g., \verb|\eqref{Eq}|) cross references instead
of ``hard'' references (e.g., \verb|(1)|). That will make it possible
to combine sections, add equations, or change the order of figures or
citations without having to go through the file line by line.

Please don't use the \verb|{eqnarray}| equation environment. Use
\verb|{align}| or \verb|{IEEEeqnarray}| instead. The \verb|{eqnarray}|
environment leaves unsightly spaces around relation symbols.

Please note that the \verb|{subequations}| environment in {\LaTeX}
will increment the main equation counter even when there are no
equation numbers displayed. If you forget that, you might write an
article in which the equation numbers skip from (17) to (20), causing
the copy editors to wonder if you've discovered a new method of
counting.

{\BibTeX} does not work by magic. It doesn't get the bibliographic
data from thin air but from .bib files. If you use {\BibTeX} to produce a
bibliography you must send the .bib files. 

{\LaTeX} can't read your mind. If you assign the same label to a
subsubsection and a table, you might find that Table I has been cross
referenced as Table IV-B3. 

{\LaTeX} does not have precognitive abilities. If you put a
\verb|\label| command before the command that updates the counter it's
supposed to be using, the label will pick up the last counter to be
cross referenced instead. In particular, a \verb|\label| command
should not go before the caption of a figure or a table.

Do not use \verb|\nonumber| inside the \verb|{array}| environment. It
will not stop equation numbers inside \verb|{array}| (there won't be
any anyway) and it might stop a wanted equation number in the
surrounding equation.

\subsection{Some Common Mistakes}\label{SCM}
\begin{itemize}
\item The word ``data'' is plural, not singular.
\item The subscript for the permeability of vacuum $\mu_{0}$, and other common scientific constants, is zero with subscript formatting, not a lowercase letter ``o''.
\item In American English, commas, semicolons, periods, question and exclamation marks are located within quotation marks only when a complete thought or name is cited, such as a title or full quotation. When quotation marks are used, instead of a bold or italic typeface, to highlight a word or phrase, punctuation should appear outside of the quotation marks. A parenthetical phrase or statement at the end of a sentence is punctuated outside of the closing parenthesis (like this). (A parenthetical sentence is punctuated within the parentheses.)
\item A graph within a graph is an ``inset'', not an ``insert''. The word alternatively is preferred to the word ``alternately'' (unless you really mean something that alternates).
\item Do not use the word ``essentially'' to mean ``approximately'' or ``effectively''.
\item In your paper title, if the words ``that uses'' can accurately replace the word ``using'', capitalize the ``u''; if not, keep using lower-cased.
\item Be aware of the different meanings of the homophones ``affect'' and ``effect'', ``complement'' and ``compliment'', ``discreet'' and ``discrete'', ``principal'' and ``principle''.
\item Do not confuse ``imply'' and ``infer''.
\item The prefix ``non'' is not a word; it should be joined to the word it modifies, usually without a hyphen.
\item There is no period after the ``et'' in the Latin abbreviation ``et al.''.
\item The abbreviation ``i.e.'' means ``that is'', and the abbreviation ``e.g.'' means ``for example''.
\end{itemize}
An excellent style manual for science writers is \cite{b7}.

\subsection{Authors and Affiliations}
\textbf{The class file is designed for, but not limited to, six authors.} A 
minimum of one author is required for all conference articles. Author names 
should be listed starting from left to right and then moving down to the 
next line. This is the author sequence that will be used in future citations 
and by indexing services. Names should not be listed in columns nor group by 
affiliation. Please keep your affiliations as succinct as possible (for 
example, do not differentiate among departments of the same organization).

\subsection{Identify the Headings}
Headings, or heads, are organizational devices that guide the reader through 
your paper. There are two types: component heads and text heads.

Component heads identify the different components of your paper and are not 
topically subordinate to each other. Examples include Acknowledgments and 
References and, for these, the correct style to use is ``Heading 5''. Use 
``figure caption'' for your Figure captions, and ``table head'' for your 
table title. Run-in heads, such as ``Abstract'', will require you to apply a 
style (in this case, italic) in addition to the style provided by the drop 
down menu to differentiate the head from the text.

Text heads organize the topics on a relational, hierarchical basis. For 
example, the paper title is the primary text head because all subsequent 
material relates and elaborates on this one topic. If there are two or more 
sub-topics, the next level head (uppercase Roman numerals) should be used 
and, conversely, if there are not at least two sub-topics, then no subheads 
should be introduced.

\subsection{Figures and Tables}
\paragraph{Positioning Figures and Tables} Place figures and tables at the top and 
bottom of columns. Avoid placing them in the middle of columns. Large 
figures and tables may span across both columns. Figure captions should be 
below the figures; table heads should appear above the tables. Insert 
figures and tables after they are cited in the text. Use the abbreviation 
``Fig.~\ref{fig}'', even at the beginning of a sentence.

\begin{table}[htbp]
\caption{Parámetros $\frac{W}{L}$ de la neurona LIF}
\begin{center}
\begin{tabular}{|c|c|c|c|}
\hline
\textbf{Table}&\multicolumn{3}{|c|}{\textbf{Table Column Head}} \\
\cline{2-4} 
\textbf{Head} & \textbf{\textit{Table column subhead}}& \textbf{\textit{Subhead}}& \textbf{\textit{Subhead}} \\
\hline
copy& More table copy$^{\mathrm{a}}$& &  \\
\hline
\multicolumn{4}{l}{$^{\mathrm{a}}$Sample of a Table footnote.}
\end{tabular}
\label{tab1}
\end{center}
\end{table}

\begin{table}[htbp]
\caption{Parámetros $\frac{W}{L}$ de la sinapsis memristiva}
\begin{center}
\begin{tabular}{|c|c|c|c|}
\hline
\textbf{Table}&\multicolumn{3}{|c|}{\textbf{Table Column Head}} \\
\cline{2-4} 
\textbf{Head} & \textbf{\textit{Table column subhead}}& \textbf{\textit{Subhead}}& \textbf{\textit{Subhead}} \\
\hline
copy& More table copy$^{\mathrm{a}}$& &  \\
\hline
\multicolumn{4}{l}{$^{\mathrm{a}}$Sample of a Table footnote.}
\end{tabular}
\label{tab1}
\end{center}
\end{table}

\begin{figure}[htbp]
\centerline{\includegraphics{fig1.png}}
\caption{Example of a figure caption.}
\label{fig}
\end{figure}

Figure Labels: Use 8 point Times New Roman for Figure labels. Use words 
rather than symbols or abbreviations when writing Figure axis labels to 
avoid confusing the reader. As an example, write the quantity 
``Magnetization'', or ``Magnetization, M'', not just ``M''. If including 
units in the label, present them within parentheses. Do not label axes only 
with units. In the example, write ``Magnetization (A/m)'' or ``Magnetization 
\{A[m(1)]\}'', not just ``A/m''. Do not label axes with a ratio of 
quantities and units. For example, write ``Temperature (K)'', not 
``Temperature/K''.

\section*{Conclusión}

Este artículo presento una implementación de neuronas LIF interconectadas entre sí, a través de un modelo de sinapsis 6T1R, cuyo componente principal era el memristor, el cual proporcionó diferentes pesos sinápticos de acuerdo con la regla de aprendizaje STDP. La construcción del sistema se hizo por medio de la herramienta de diseño esquemático Xschem y la simulación con NGSPICE. Cabe señalar que se utilizó la tecnología de SKY130, desarrollado por SkyWater Technology en colaboración con Google, que es un proceso de fabricación de semiconductores que ha sido puesto a disposición de la comunidad de diseño de circuitos de manera abierta y gratuita. Esto facilita la innovación y reduce significativamente los costos asociados con el diseño y la producción de chips, haciendo más accesible la experimentación y el desarrollo de nuevas tecnologías. Dentro de los elementos proporcionados por este nodo tecnológico, se encuentra el memristor, una tecnología clave que facilita la implementación de sinapsis artificiales en sistemas neuromórficos, soportando redes neuronales pulsantes y el aprendizaje STDP. Su capacidad para cambiar de estado de manera no volátil y su integración en la tecnología CMOS permiten la creación de sistemas de cómputo avanzados que no solo son más eficientes energéticamente, sino también más apto de emular las capacidades adaptativas y de aprendizaje del cerebro humano. La convergencia de estas tecnologías promete revolucionar el campo de la inteligencia artificial, llevando el rendimiento y la eficiencia del cómputo a nuevos niveles.

\bibliographystyle{IEEEtranN}
\bibliography{referencias.bib}

\end{document}
