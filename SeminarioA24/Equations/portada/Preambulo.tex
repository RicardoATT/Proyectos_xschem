\documentclass[letterpaper, 12pt]{article} %10pt por default
%\usepackage[paper=a4paper,left=24mm,right=24mm,top=20mm,bottom=20mm]{geometry}

\usepackage[utf8]{inputenc}
\usepackage[spanish]{babel}
\usepackage{geometry}
\usepackage{lipsum} %por si hay que hacer pruebas de como se ver\'ia
\usepackage{graphicx}
%\usepackage{rotating}
\usepackage{paracol}
\usepackage{multicol}
\usepackage{setspace}
\setstretch{1.2}
\setlength{\parskip}{16pt}
\usepackage{xstring}
\usepackage{xcolor}
\usepackage{xargs}
\usepackage{bbding} %para las palomitas
\usepackage{ragged2e}
\usepackage[hidelinks]{hyperref}
\usepackage{wrapfig}
\usepackage{lscape}
\usepackage{rotating}
\usepackage{epstopdf}
\usepackage{amsmath}
\usepackage{amssymb}
%\usepackage{caption}
%\usepackage{subcaption}
\usepackage{subfig}
%\usepackage[ruled,linesnumbered]{algorithm2e}
\usepackage{listings}
%\usepackage[toc, acronym]{glossaries}
\usepackage{verbatim}
%\usepackage{natbib}
\usepackage{float}
\usepackage{enumitem}
\usepackage{mathrsfs}
\usepackage{tcolorbox}
\usepackage[export]{adjustbox}
\usepackage[makeroom]{cancel}
\usepackage[firstpage=true]{background}
\usepackage[numbers]{natbib}

\usepackage{tikz}
\usetikzlibrary{positioning,shadows,backgrounds,automata,arrows,arrows.meta}
%\usetikzlibrary{arrows} %arrows meta
%\usepackage{tikz-qtree} % Easy tree drawing tool
%\tikzset{every tree node/.style={align=center,anchor=north}, level distance=2cm} % Configuration for q-trees


%\usepackage[breakable]{tcolorbox}



%Para estilo fancy
%\usepackage{fancyhdr}
%\usepackage{lastpage}fenter
%\pagestyle{fancy}
%\fancyhf{}

%\renewcommand{\headrulewidth}{0pt} %Quitar la l\'inea del encabezado de fancy
%\rfoot{\footnotesize{ P\'agina \thepage \hspace{1pt} de \pageref{LastPage} }}





\decimalpoint
%If you use either the babel or polyglossia package you'll have to change the name for the particular language
%you use with babel or polyglossia.
\addto\captionsspanish{% Replace "english" with the language you use
	\renewcommand{\contentsname}{Tabla de contenido}%
	\renewcommand{\listfigurename}{Lista de figuras}
	\renewcommand{\listtablename}{Lista de tablas}
	\renewcommand{\lstlistlistingname}{Lista de programas}
	\renewcommand{\glossaryname}{Glosario}
	% NOMBRES DE FIGURA, TABLA, ETC.
	%\renewcommand{\thefigure}{Fig.(\arabic{figure})}
	\renewcommand{\figurename}{Figura}
	%\renewcommand{\thetable}{Tab.(\arabic{table})}
	\renewcommand{\tablename}{Tabla}
	%\renewcommand{\theequation}{Eq.(\arabic{equation})}
	\renewcommand{\lstlistingname}{Programa}
}
% RENOMBRAR BIBLIOGRAF\'IA
%\renewcommand{\bibname}{\section{Referencias}}
%\renewcommand{\theenumi}{\Alph{enumi}}
\newcommand{\reftab}[1]{Tab.(\ref{#1})}
\newcommand{\reffig}[1]{Fig.(\ref{#1})}
\newcommand{\refeq}[1]{Ec.(\ref{#1})}
\newcommand{\refexp}[1]{Exp.(\ref{#1})}
\newcommand{\refdef}[1]{Def.(\ref{#1})}
\newcommand{\refsec}[1]{Sec.(\ref{#1})}
\newcommand{\refssec}[1]{Subsec.(\ref{#1})}
\newcommand{\refej}[1]{Ej.(\ref{#1})}
\newcommand{\refprog}[1]{Prog.(\ref{#1})}


% SIEMRPE EXISTIR\'A UNA CARPETA IMG
\graphicspath{{./img/}}

% ESPACIO ENTRE COLUMNAS DE MULTICOL
\setlength{\columnsep}{1cm}

% CONFIGURACI\'ON DE HIPERV\'INCULOS
%\hypersetup{
%    colorlinks=true,
%    linkcolor=black,
%    urlcolor=magenta
%}

% TAMA\~NO DE LA HOJA
\geometry{
	%headsep= 0pt,
	%head=0pt,
	%ignoreall,
	hmargin= {2.5cm, 2.5
		cm}, %izquierda, derecha
	vmargin= {2.5cm, 2.5cm} %arriba, abajo
}


% PREGUNTA Y RESPUESTA (pregunta dentro de itemizador)
\newcommand{\preg}[1][Pregunta]{
 \item \textbf{#1} \\
}
\newcommand{\resp}[1][Respuesta]{
 \textit{#1} \ \\
}

% ELEMENTO LIBRE DE ITEMIZADOR
\newcommand{\itemlibre}[1]{
 \tab - #1. \\
}

% CONCEPTO DE GLOSARIO: 1->llave, 2->concepto, 3->descripcion
% Requiere libreria glossaries
\newcommand{\itemg}[2]{
 \item \textbf{#1} #2 
 
 \enter
 
}

% ESPACIADORES
\newcommand{\enter}{\vspace{0.5cm}}
\newcommand{\tab}{\hspace{1cm}}

% L\'INEA PARA LLENADO (param: medida de la l\'inea en mm)
\newcommand{\sublinea}[1]{
 \rule{#1mm}{0.1mm}
<}

% LEYENDAS DE IMAGEN ALTERADAS A CURSIVA
\newcommand{\captionit}[1]{
    \caption{\textit{#1}}
}
\newcommand{\subcaptionit}[1]{
    \subcaption{\textit{#1}}
}

% CUADRO DE REMARCACI\'ON
\newenvironment{remark}[1]{
	\begin{tcolorbox}[
		colback= myNaranja!25,
		colframe= blue254!75,
		title=#1,
		arc= 3mm,
		sharp corners= northwest
	]
		\fontfamily{gag}\selectfont
}{\end{tcolorbox}}

\newcounter{cteorema}
\newenvironment{theo}[1]{
	\addtocounter{cteorema}{1}
	\begin{tcolorbox}[
		colback=orange!5,
		colframe=blue!50!black,
		title= \textbf{Teorema \thesection.\arabic{cteorema}. #1},
		arc= 3mm,
		sharp corners= north
		]
		\fontfamily{gag}\selectfont
}{\end{tcolorbox}}

\newcounter{cejemplo}
\newenvironment{ejem}[1]{
	\refstepcounter{cejemplo}
	\begin{tcolorbox}[
		breakable,
		colback=blue!5,
		colframe=blue!75!black,
		title= \textbf{Ejemplo \arabic{cejemplo}. #1},
		arc= 3mm,
		sharp corners= all
		]
		\fontfamily{gag}\selectfont
}{\end{tcolorbox}}

\newcounter{cdefinicion}
\newenvironment{defon}[1]{
	\addtocounter{cdefinicion}{1}
	\begin{tcolorbox}[
		colback=orange!20,
		colframe=blue254!75,
		title= \textbf{Definici\'on \thesection.\arabic{cdefinicion}. #1},
		arc= 3mm,
		sharp corners= west
		%breakable,
		%enhanced
		]
		\fontfamily{gag}\selectfont
	}{\end{tcolorbox}}



% CONVERSI\'ON DE CONTADORES A GLOBALES
%Para hacer global el contador de figura y no se repita al hacer un paracol
\globalcounter{figure}
\globalcounter{equation}








% COLORES DE PORTADA
\definecolor{myAzul}{HTML}{234ECA}
\definecolor{blue254}{HTML}{02528F}
\definecolor{myVerde}{HTML}{36A736}
\definecolor{myNaranja}{HTML}{FF4312}
\definecolor{guinda}{HTML}{660000}
\definecolor{azul}{HTML}{1A079F}
\definecolor{negro}{HTML}{000000}
\definecolor{blanco}{HTML}{FFFFFF}
\definecolor{dorado}{HTML}{996515}
\definecolor{ududff}{rgb}{0.30196078431372547,0.30196078431372547,1}
\definecolor{template_blue}{HTML}{003473}   % Color of Leiden University logo (Lei-Blauw)
\definecolor{subtitle}{cmyk}{0,0,0,0}       % Color for subtitle (white)
\definecolor{template_text}{HTML}{434655}   % Color for text
\definecolor{template_lightgrey}{HTML}{A8AABC}  % Color of the chapter banner   


% BORRAR SANGR\'IA EN GENERAL
\setlength\parindent{0pt}

% ALINEAR A LA DERECHA LOS N\'UMEROS DE FOOTNOTE
%\footnotemargin{\@makefnmark\hss}




% ESTILO PARA LISTINGS
\definecolor{codegreen}{rgb}{0,0.6,0}
\definecolor{codegray}{rgb}{0.5,0.5,0.5}
\definecolor{codepurple}{rgb}{0.58,0,0.82}
\definecolor{backcolour}{rgb}{0.95,0.95,0.92}

\lstdefinestyle{mystyle}{
	backgroundcolor=\color{backcolour},   
	commentstyle=\color{codegreen},
	keywordstyle=\color{blue},
	numberstyle=\tiny\color{codegray},
	stringstyle=\color{codepurple},
	basicstyle=\ttfamily\footnotesize,
	breakatwhitespace=false,         
	breaklines=true,                 
	captionpos=b,                    
	keepspaces=true,                 
	numbers=left,                    
	numbersep=5pt,                  
	showspaces=false,                
	showstringspaces=false,
	showtabs=false,                  
	tabsize=1
}

\lstset{style=mystyle}




\makeatletter
\newcommand\binomialCoefficient[2]{%
	% Store values 
	\c@pgf@counta=#1% n
	\c@pgf@countb=#2% k
	%
	% Take advantage of symmetry if k > n - k
	\c@pgf@countc=\c@pgf@counta%
	\advance\c@pgf@countc by-\c@pgf@countb%
	\ifnum\c@pgf@countb>\c@pgf@countc%
	\c@pgf@countb=\c@pgf@countc%
	\fi%
	%
	% Recursively compute the coefficients
	\c@pgf@countc=1% will hold the result
	\c@pgf@countd=0% counter
	\pgfmathloop% c -> c*(n-i)/(i+1) for i=0,...,k-1
	\ifnum\c@pgf@countd<\c@pgf@countb%
	\multiply\c@pgf@countc by\c@pgf@counta%
	\advance\c@pgf@counta by-1%
	\advance\c@pgf@countd by1%
	\divide\c@pgf@countc by\c@pgf@countd%
	\repeatpgfmathloop%
	\the\c@pgf@countc%
}
\makeatother


