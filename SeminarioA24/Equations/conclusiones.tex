\section{Conclusiones}
En conclusión, se implementaron los 2 circuitos en lenguaje Verilog de manera exitosa.

Para el desplazador de barril de 4 bits, se comprendió como es que se implementa este tipo de circuito combinatorio y su importancia en varias aplicaciones como lo son el procesamiento de señales, la criptografía y sobre todo en la arquitectura de microprocesadores.

Para el flip flop tipo T, se implementó de manera correcta y se diferencio su funcionamiento contra un tipo de flip flop implementado anteriormente (flip flop tipo D).

Se comprobó el funcionamiento de los circuitos utilizando las simulaciones de forma de onda en ModelSim y con el visor RTL se analizó como es que la herramienta de Quartus implementa los dispositivos.

En los Anexos se pueden encontrar los códigos implementados junto con sus respectivos bancos de pruebas.